\documentclass[]{article}
\usepackage[english]{babel}
\usepackage{booktabs}
\usepackage{multirow}
\title{Advanced Nuclear Reactor Design Report}
\author{Matthew Eklund, Christopher Morrison, Jaron Senecal}
\date{Spring 2016}

\begin{document}
\maketitle

\section{Introduction}

With the increasing need for reliable energy sources that contribute minimal quantities of greenhouse gases, nuclear power remains a valuable option as compared to traditional fossil fuel sources.  
For geographical locations with high power requirements and excessive pollution, nuclear energy is a particularly attractive solution as it contributes nearly zero carbon emissions to the atmosphere.  
Nuclear power does not depend on the weather or the amount of available sunlight, the fuel is relatively cheap, and the technologies have been employed for thousands of operating years (taking into account all the reactors throughout the world which have operated in over the past 50 years) with only several major incidents on record.

There are many nuclear reactor designs existent throughout the world in the year 2016.  
Many designs have been successful in their deployment, others have had limited success, and others have not surpassed the research and design phase to the prototyping and development phases.  
While many of the existent technologies are well understood for many years, some are still untested.  
The information is provided in a wide range of resources from technical journals to textbooks, internal industrial and government documents, etc.  
There is no lack in the amount of information available for students in Nuclear Engineering to aid in designing nuclear reactors.  
However, the current academic environment in the US typically focuses on understanding the physical processes that occur within a nuclear reactor, the standard reactor types used today and in the past, and in understanding codes used for reactor analysis.  

This report documents the goals of the work to be performed for this independent study throughout the semester including the planned research milestones and the end deliverables.

\section{Motivation}

Many reactor designs have been created and built in the last century and were conceptualized to satisfy certain requirements and the technology available at the time.
The authors have found a lack of a centralized source of information for designing a nuclear reactor from the ground up using new techniques and improving technologies.
With design requirements and technology capabilities and availability continuing to increase, there exists a need to define principles by which a nuclear engineer may take advantage of these advances to solve the energy needs of society.
The motivation for this project is to document the methods and process by which a nuclear engineer may design a nuclear reactor to satisfy the given energy and construction requirements.  
As a part of the work done toward designing a nuclear reactor optimized for a specific mission, additional documentation toward creating a textbook in Advanced Nuclear Reactor Design Principles will be written throughout the semester.

\section{Mission}

Nuclear reactors are employed to fulfill several purposes.  The major geographical locations of their deployment along with their purposes and users/customers are detailed in the following table.

\begin{center}
\begin{tabular}[pos{c}]{| c | c | c |}
\hline
Geographical Location & Purpose & Customer/Application \\
\hline
Land & Electrical Power Production & Industry/Government \\
\hline
Land & Science and Medical Research & Industry/Government \\
\hline
Space & Energy and Propulsion & Industry/Government/Military \\
\hline
Sea & Energy and Propulsion & Government/Military \\
\hline
\end{tabular}
\end{center}

The authors determined to approach designing a land-based reactor which operates within a power production facility.
The reason for designing a reactor for this deployment is that it is the most widely-used platform for building nuclear power throughout the world.
As developing countries seek carbon-clean energy resources with a sufficient power density to provide electricity for their growing economies, nuclear energy remains a powerful and attractive alternative to pollution-producing plants which burn on traditional fossil fuels.

With the function and geographical location for this reactor decided, the mission which drives the design of a particular type of reactor was explored.  Each nuclear reactor must take into account (to varying degrees) the contemporary atmosphere in which it will be built; this includes factors such as current and potential future technologies, the economic and political environment, economic feasibility, etc.
In beginning this project, the authors brainstormed on important problems related to the world's energy crisis and nuclear power of today.  The table below includes the major problems chosen by the team which should be addressed when designing a new power-generation facility.

\begin{center}
\begin{tabular}[c]{| p{5cm} | c |}
\hline
Problem & Ranking of Importance \\
\hline
High cost of constructing and \mbox{operating} a reactor & \multirow{2}{*}{3} \\
\hline
Buildup of nuclear waste (long-lived actinides and fission products) & \multirow{3}{*}{1} \\
\hline
Dwindling fissile material \mbox{available} & \multirow{2}{*}{5} \\
\hline
Lack of passive safety systems in reactors & \multirow{2}{*}{2} \\
\hline
Lengthy construction times for nuclear facilities & \multirow{2}{*}{4} \\
\hline
Proliferation concerns & \multirow{1}{*}{6} \\
\hline
\end{tabular}
\end{center}

The justification for choosing the ranking system is based on the current political and economical status of nuclear power in the world.  
The costs of building and operating a reactor can only be reduced by a finite quantity as compared to the total cost.  
As for the price and availability of fissile nuclear material, this is of little concern.
At the moment, triuranium octoxide, or "yellow cake," costs only \$34.65 USD and is plentiful in supply.  

Nuclear reactor facilities take anywhere from several years to over a decade to build and will always be a long-term investment for each plant.  
The concern over proliferation (i.e. the use of nuclear material by malign/terrorist organizations to produce weapons) will always exist and can be prevented through increased plant defenses, security measures, emergency plans, etc.  
The problem of increasing nuclear waste and increasing passive safety systems of this newly-designed nuclear reactor were deemed of highest importance.

Following the brainstorming session and group discussions, the mission for the reactor was determined.
The mission for the new reactor design is meant to provide an economic solution to the production of long-lived radioactive waste produced by standard light water reactors (LWRs).  The official mission of the project is thus:
\begin{itemize} 
  \item To design a nuclear power reactor which is optimized to burn long-lived actinides while implementing additional passive safety features over standard LWR designs.
\end{itemize}

The process by which this reactor is designed is documented in this report.

\section{Desirables}

\end{document}