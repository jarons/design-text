\documentclass[]{report}
\usepackage[english]{babel}
\usepackage{longtable}
\usepackage{booktabs}
\usepackage{multirow}
\usepackage{multicol}
\title{Advanced Nuclear Reactor Design Report}
\author{Matthew Eklund, Christopher Morrison, Jaron Senecal}
\date{Spring 2016}

% Smaller margins, fix these later:
\addtolength{\oddsidemargin}{-0.75in}
\addtolength{\evensidemargin}{-1.25in}
\addtolength{\textwidth}{1.75in}%
%\addtolength{\topmargin}{-1in}
\addtolength{\textheight}{1.5in}

\begin{document}
\maketitle

\section{Introduction}

With the increasing need for reliable energy sources that contribute minimal quantities of greenhouse gases, nuclear power remains a valuable option as compared to traditional fossil fuel sources.  
For geographical locations with high power requirements and excessive pollution, nuclear energy is a particularly attractive solution as it contributes nearly zero carbon emissions to the atmosphere.  
Nuclear power does not depend on the weather or the amount of available sunlight, the fuel is relatively cheap, and the technologies have been employed for thousands of reactor-years %operating years (taking into account all the reactors throughout the world which have operated in over the past 50 years) 
with only several major incidents on record.

There are many nuclear reactor designs existent throughout the world in the year 2016.  
Many designs have been successful in their deployment, others have had limited success, and others have not surpassed the research and design phase to the prototyping and development phases.  
While many of the existent technologies are well understood for many years, some are still untested.  
The information is provided in a wide range of resources from technical journals to textbooks, internal industrial and government documents, etc.  
There is no lack in the amount of information available for students in Nuclear Engineering to aid in designing nuclear reactors.  
However, the current academic environment in the US typically focuses on understanding the physical processes that occur within a nuclear reactor, the standard reactor types used today and in the past, and in understanding codes used for reactor analysis.  

This report documents the goals of the work to be performed for this independent study throughout the semester including the planned research milestones and the end deliverables.

\section{Motivation}

Many reactor designs have been created and built in the last century and were conceptualized to satisfy certain requirements and the technology available at the time.
The authors have found a lack of a centralized source of information for designing a nuclear reactor from the ground up using new techniques and improving technologies.
With design requirements and technology capabilities and plant availability continuing to increase, there exists a need to define principles by which a nuclear engineer may take advantage of these advances to solve the energy needs of society.
The motivation for this project is to document the methods and process by which a nuclear engineer may design a nuclear reactor to satisfy the given energy and construction requirements.  
As a part of the work done toward designing a nuclear reactor optimized for a specific mission, additional documentation toward creating a textbook in Advanced Nuclear Reactor Design Principles will be written throughout the semester.

\section{Mission}

Nuclear reactors are employed to fulfill several purposes.  The major geographical locations of their deployment along with their purposes and users/customers are detailed in the following table.

\begin{center}
\begin{tabular}[pos{c}]{| c | c | c |}
\hline
Geographical Location & Purpose & Customer/Application \\
\hline
Land & Electrical Power Production & Industry/Government \\
\hline
Land & Science and Medical Research & Industry/Government \\
\hline
Space & Energy and Propulsion & Industry/Government/Military \\
\hline
Sea & Energy and Propulsion & Government/Military \\
\hline
\end{tabular}
\end{center}

The authors determined to approach designing a land-based reactor which operates within a power production facility.
The reason for designing a reactor for this deployment is that it is the most widely-used platform for building nuclear power throughout the world.
As developing countries seek carbon-clean energy resources with a sufficient power density to provide electricity for their growing economies, nuclear energy is the affordable and dispatchable %remains a powerful and attractive 
alternative to pollution-producing plants which burn traditional fossil fuels.

With the function and geographical location for this reactor decided, the mission which drives the design of a particular type of reactor was explored.  Each nuclear reactor must take into account (to varying degrees) the contemporary atmosphere in which it will be built; this includes factors such as current and potential future technologies, the economic and political environment, economic feasibility, etc.
In beginning this project, the authors brainstormed on important problems related to the world's energy crisis and nuclear power of today.  %The table below includes the major problems chosen by the team which should be addressed when designing a new power-generation facility.
The major problems selected by the team to be addressed when designing a new power-generation facility are listed in order of priority
\begin{itemize}
  \item[1.] Buildup of nuclear waste (long-lived actinides and fission products)
  \item[2.] Lack of passive safety systems in reactors
  \item[3.] High cost of constructing and operating a reactor
  \item[4.] Lengthy construction times for nuclear facilities, including lead time for research and development
  \item[5.] Dwindling fissile material available
  \item[6.] Proliferation concerns
\end{itemize}
Of course there are other non-negotiable priorities that will be considered as constraints, such as maintaining an acceptable level of safety.

\iffalse
\begin{center}
\begin{tabular}[c]{| p{5cm} | c |}
\hline
Problem & Ranking of Importance \\
\hline
High cost of constructing and \mbox{operating} a reactor & \multirow{2}{*}{3} \\
\hline
Buildup of nuclear waste (long-lived actinides and fission products) & \multirow{3}{*}{1} \\
\hline
Dwindling fissile material \mbox{available} & \multirow{2}{*}{5} \\
\hline
Lack of passive safety systems in reactors & \multirow{2}{*}{2} \\
\hline
Lengthy construction times for nuclear facilities & \multirow{2}{*}{4} \\
\hline
Proliferation concerns & \multirow{1}{*}{6} \\
\hline
\end{tabular}
\end{center}
\fi

The justification for choosing the ranking system is based on the current political and economical status of nuclear power in the world.  
The costs of building and operating a reactor can only be reduced by a finite quantity as compared to the total cost.  
As for the price and availability of fissile nuclear material, this is of little concern.
At the moment, triuranium octoxide, or "yellow cake," costs only \$34.65 USD and is plentiful in supply.  

Nuclear reactor facilities take anywhere from several years to over a decade to build and will always be a long-term investment for each plant.  
The concern over proliferation (i.e. the use of nuclear material by malign/terrorist organizations to produce weapons) will always exist and can be prevented through increased plant defenses, security measures, emergency plans, etc.  
The problem of increasing nuclear waste and increasing passive safety systems of this newly-designed nuclear reactor were deemed of highest importance.

Following the brainstorming session and group discussions, the mission for the reactor was determined.
The mission for the new reactor design is meant to provide an economic solution to the production of long-lived radioactive waste produced by standard light water reactors (LWRs).  The official mission of the project is thus:
\begin{itemize} 
  \item To design a nuclear power reactor which is optimized to burn long-lived actinides while implementing additional passive safety features over standard reactor designs.
\end{itemize}

The process by which this reactor is designed is documented in this report.

\section{Design Attributes}
With the mission for designing the new reactor having been decided, the more specific details of the desired reactor functions and attributes must be determined.
For a reactor that is optimized for burning long-lived actinides, certain attributes are essential.
The list of most important attributes for this reactor is as follows:

\begin{itemize} 
  \item The new reactor design must maintain and exceed the safety levels achieved by modern reactor designs.
  \item The reactor must provide a means for burning long-lived actinides.  This will require higher-energy neutrons to increase fission rate of actinides.
  \item The reactor design should take advantage of natural processes/phenomena to increase safety wherever possible.  Relying on natural processes will decrease the need for engineered systems for increased safety and reliability.  The simplified design may also reduce costs in construction and maintenance.
  \item As far as possible, decisions should favor an economical design by reducing the cost in construction and operation.
\end{itemize}

Along with these primary and most essential attributes, there are several secondary attributes that are less crucial to achieve in the final design.  The list of secondary attributes is as follows:
\begin{itemize}
  \item The burner reactor design may be considered a "stepping stone" toward a breeder reactor design.  Modular or replaceable features which may facilitate a breeding environment within the core may be considered.
  \item If possible, technologies for improving proliferation resistance in the reactor design should be incorporated.
  \item If possible, the reactor should be designed using technologies which are available either currently or within a reasonable amount of time so as to reduce the time necessary for construction and implementation of the reactor (including research and development phases).
\end{itemize}

\section{Core Component Desirables and Potential Materials}

At this point, no materials, geometries, etc. have been disqualified from potential burner reactor designs for this project;  the authors wish to keep an open-minded mentality to this design process.
In order to determine which materials would be optimal for a burner reactor, a brainstorming session resulted in a list of common and potential materials for arguably the three most important components of a nuclear reactor core: the fuel, the fuel cladding, and the coolant (no moderator is required because this will be a fast reactor). 
 The fuel is what contains the fissile or fissionable material to produce fission nuclear reactions to produce heat. It needs to withstand an extreme environment while conducting heat to the working fluid. %(which is the whole purpose of designing and building a nuclear reactor core).  
 The cladding exists to contain the fuel and is the first line of defense for the release of fission products from the core.  Finally, the coolant is what extracts the heat from the fuel to eventually be used in a heat generation cycle (e.g., Rankine or Brayton) to produce electricity.
Found below is a table of the desirable attributes for the fuel, cladding, and coolant specifically for a burner reactor as determined by the authors.  They are not listed in order of priority.



\begin{center}
\begin{tabular}[c]{| c | p{2cm} | p{2cm} | p{2.0cm} |}
\hline
\multicolumn{4}{|c|}{Desirable Component Attributes} \\
\hline
 & Fuel & Cladding & Coolant \\
\hline
\hline
\multirow{2}{*}{Thermophysical} & High Density & &\\
\cline{2-4}
 & \multicolumn{3}{|c|}{High Thermal Conductivity} \\
\cline{2-4}
 & \multicolumn{2}{|c|}{High Melting Point} & Low Melting Point \\
\cline{2-4}
\multirow{2}{*}{Properties} & & & High Specific Heat \\
\cline{2-4}
 & & & High Boiling Point (if liquid) \\
\hline
\multirow{3}{*}{Neutronics/Reactivity} & \multicolumn{3}{|c|}{Low Neutron Capture Cross Section} \\
\cline{2-4}
 & \multicolumn{3}{|c|}{Low Neutron Moderation} \\
\cline{2-4}
 & High Reactivity & & \\
\hline
\multirow{3}{*}{Mechanical/} & \multicolumn{3}{|c|}{Chemical Compatibility}  \\ % at Interfaces
\cline{2-4}
 & \multicolumn{3}{|c|}{Manufacturable and Low Cost} \\
\cline{2-4}
\multirow{2}{*}{Chemical Properties} & \multicolumn{2}{|c|}{Ductile} & Optical Transparency \\ %Brittle/
\cline{2-4}
 & Low Swelling / Growth & High Strength & Easy Pumping \\
\cline{2-4}
 &  & Retains Fission Products &  \\
\hline
\end{tabular}
\end{center}

For these three major components, the list of desirable attributes can be used to determine a set of compatible materials.  After a brainstorming session, the table below includes the major choices of materials that would be suitable for each component.

\begin{center}
\begin{tabular}[c]{| p{2.0cm} | p{2.5cm} | p{2.5cm} |}
\hline
\multicolumn{3}{|c|}{Potential Component Materials} \\
\hline
Fuel & Cladding & Coolant \\
\hline
Uranium dioxide (UO$_2$) & Zirconium alloys & Sodium (Na) \\
\hline
Uranium carbide (UC) & Steels & Sodium potassium (NaK) \\
\hline
Uranium nitride (UN) & Iron-chromium-aluminum (FeCrAl) alloy & Lead (Pb) \\
\hline
Uranium-zirconium (UZr) alloys & Nickel (Ni) alloys & Lead-Bismuth Eutectic\\
\hline
Uranium silicide (U$_3$Si$_2$)  & Chromium (Cr) alloys & Nitrogen (N$_2$)\\ %Particle fluid
\hline
Plutonium-Uranium compounds  & Silicon carbide (SiC) & Fluoride Salts \\
\hline
Salt Fuels & Ceramic Composites & Chloride Salts \\
\hline
Liquid Fuels & Hastelloys & Noble Gases (e.g., Argon, Helium) \\
\hline
 & & Mercury (Hg)  \\
\hline
\end{tabular}
\end{center}
% I'm taking out MOX and mixed nitrides, trying to make it more generic

All of these and more materials could potentially serve as fuel, cladding, coolant, and other components (structural, piping, etc.) for the reactor core.  However, the goal is to achieve an optimal design to take advantage of all available materials for accomplishing the mission of the reactor's construction.  With seemingly endless options, a method for narrowing down the list of potential materials is needed. For each of the reactor core components materials with less desirable properties must be eliminated.  

One possible method of narrowing material selection choices is to create a grading matrix.  Criteria which are crucial for particular reactor component functions will have a rating given to each material; a "1" indicates the material is nearly perfectly suitable for such a function, and "5" indicates the material will not function well for this purpose.

A notable limitation of this grading rubric is that certain material properties may be perfectly suitable for a reactor component function, and increasing this value in the material will produce no further advantage.  This must be taken into account when creating the rubric; for example, the melting point for a coolant typically is recommended to be at room temperature or below, but having a melting temperature at temperatures incredibly lower than this may not be useful.

The grading matrix is broken into three parts for the fuel, cladding, and coolant separately.  The authors brainstormed and included a grading based on technical data which is included (where available) and otherwise determined through recommendations of previous reactor designers and analysts, operating reactor experience, etc.

% We need to add the columns: swelling, R\&D needs, (max burn up)
\begin{center}
\hspace*{-2cm} %this is tacky and dangerous
\begin{tabular}[c]{| p{1.6cm} | p{1.2cm} | p{2.05cm} | p{1.5cm} | p{1.5cm} | p{2.0cm} | p{1.5cm} | p{1.5cm} | p{2cm} |}
\hline
\multicolumn{9}{|c|}{Potential Fuel Materials Grading Matrix} \\
\hline
 & Physical Density \{g/cm$^3$\} & Fission Material Density \{g/cm$^3$\} & Thermal Conductivity  \{W/(m-K)\} & Melting Point \{K\} & Reactivity at.\%inert$\times \sigma_{\gamma}$ \{barn\} & Chemical Compatibility & Ductility & Linear Expansion Coefficient \{10$^6$(1/K)\} \\
\hline
Uranium dioxide (UO$_2$) & \{10.963\} & \{9.664\} & 5 \{2.6\} (1523 K, theor. density) & 1 \{3120\} & 2 \{2.48E-3\} & not sodium & & \{9.8\} (300 K)  \\
\hline
Uranium carbide (UC) & \{13.630\} & \{12.970\} & \{23.0\} (700 K) & \{2793\} & 1 \{1.09E-3\} & & & \{10.5\} (300 K)\\
\hline
Uranium nitride (UN) & \{14.420\} & \{13.619\} & \{20.9\} (1000 K) & 1 \{3123\} & 1 \{1.90E-3\} & & & \{7.5\} (300 K) \\
\hline
%Uranium zirconium (UZr) alloys & \\
U-20\%Pu-10\%Zr & & \{14.1\} & \{16\} & \{1400\} & 4 \{3.12E-1\} &  & & \{17\}\\
%http://www.sciencedirect.com/science/article/pii/S1738573315000753
\hline
Uranium silicide (U$_3$Si$_2$)  & \{12.2\} & \{11.3\} & \{22\} (1000 K) & \{1938\} & \{8.24E-3\} & & & \{16.1\}\\
\hline
%Plutonium (Pu)  & \{19.840\} & \{19.840\} & \{6.5\} (508 K) & \{913\} & & & & \{33.9\} (400-470 K) \\
% We really shouldn't use Pu, 
%\hline
%Mixed Nitrides  &  \\
%\hline
Salt Fuels &  \\
\hline
Liquid Fuels &  \\
\hline
% Mixed Oxides (MOX)

\end{tabular}
\end{center}
% Maybe we should have a table for things with numbers and a second table for numbers and qualities that are harder to determine. If they aren't good in the first table, they aren't worth exploring in the second.

From this comparison we see that Uranium Nitride and the Uranium metal alloy are likely to be quite favorable. The uranium metal alloy has been proven in EBR-II. Note that these results are not final, later in the design process we may change our priorities and select a different material for investigation.

Next we examine several potential coolants. Several properties and figures of merit are listed in Table~\ref{tab:coolants}. For the figures of merit, lower numbers are better. The turbulent, forced convection figure of merit is calculated from the thermo-physical properties as $FOM=\left(\frac{\mu^2}{\beta\rho^2c_p^{1.8} } \right)^{0.36}$. The Natural Circulation figure of merit is $FOM= \left( \frac{\mu^2}{\beta\rho^2c_p^{1.8}}  \right)^{0.36}$. The heat exchanger figure of merit is given by $FOM= \frac{\mu^{0.2}}{\rho^{0.3}c_p^{0.6}k^{0.6}}$

\begin{table} \label{tab:coolants}
\hspace*{-2cm} %this is tacky and dangerous
\begin{tabular}{|l|c|c|c|c|c|c|}
%
% We need to include: activation (problems for maintenance); chemical stability (needs cover gas?);
%
\centering
%\hline
Coolant & Melting Temp, $^{\circ}$C & Boiling Temp, $^{\circ}$C & $\sigma_\gamma $ at 0.1MeV, $10^-5$ b & Pump FOM & Nat Circ FOM & HX FOM \\
\hline
Sodium & 97.8 & 882.8 & 1.32 & 0.137 & 9.93 & 0.0135 \\
\hline
NaK (23\%-77\%) & -12.6 & 784 & 152 & 0.456 & 12.9 & 0.0299\\
\hline
Lead & 327.4 & 1745 & $\sim$ 60 & 0.560 & 11.4 & 0.0821\\
\hline
Pb-Bi (44.5\%-55.5\%) & 125 & 1638 & $\sim$ 400 & 0.573 & 9.80 & 0.0870\\
\hline
FLiNaK & 454 & 1570 & 2.541 & 0.013 & 4.57 & 0.182\\
\hline
\end{tabular}
\caption{Material property comparison for potential liquid coolants.}
\end{table}

Next we'll consider gaseous coolants. Gas coolants have poorer heat transfer properties, but they are easier to connect to power conversion systems. Gases naturally tolerate higher temperatures, which facilitates higher thermal efficiencies. However, that efficiency is diminished by additional pumping power requirements.
For gases like Helium, leakage can be a significant concern. 

\begin{table} \label{tab:coolants}
\hspace*{-2cm} %this is tacky and dangerous
\begin{tabular}{|l|c|c|c|c|c|c|}
%
% We need to include: leakage issues; chemical stability;
%
% Probably don't need: Boiling Temp; 
%
\centering
%\hline
Coolant &  $\sigma_a$ at 0.1MeV,  & Pump FOM & Nat Circ FOM & HX FOM & Corrosion Issues & Leakage\\
 & $10^-5$ b & & & & &\\
\hline
Helium &  & & & & Diffusion into clad/structures? & highest\\
\hline
CO$_2$ & & & & & Carbrization? &\\
\hline
Air & \\
\hline
N$_2$ &\\
\hline
Argon & &\\
\hline
\end{tabular}
\caption{Material property comparison for potential gas coolants.}
\end{table}


\end{document}