\chapter{Learning from Other Designs}
In the process of learning about nuclear engineering one undoubtedly learns a great deal about existing reactor designs. On one hand the designer should start a project with a clean slate, but on the other hand knowledge in memory can greatly accelerate the design process.
Trends, heuristics, and limitations are useful as long as the designer does not prematurely eliminate design possibilities.

In this chapter we will discuss types of nuclear reactors that have been built. The strengths and shortcomings will be listed, although not exhaustively. 
The purpose of this chapter is not to promote one design over another, because different constraints, priorities, and goals will favor different designs.

\section{Light Water Reactors}
Light water reactors are by far the most common reactor in the world. 
\begin{table}[!h]
\begin{tabular}{c|c}
  Advantages & Disadvantages \\
  \hline
  Widely used technology &  \\
  Inexpensive coolant & Corrosion issues \\
   & Requires enriched fuel \\
%  \hline
\end{tabular}
\end{table}

\subsection{Pressurized Water Reactors}
Pressurized water reactors are the more common type of light water reactor. The water must be kept high pressure to preclude boiling inside the primary coolant loop.

\subsection{Boiling Water Reactors}
Boiling Water Reactors aim for improved efficiency by using a direct power conversion cycle. However, boiling in the reactor core leads to a variety of challenges. 
\begin{table}[!h]
\begin{tabular}{c|c}
  Advantages & Disadvantages \\
  \hline
  Improved Efficiency & Two-phase flow \\
  Load following & Unstable at low power\\
   & Radioactivity in the steam turbine\\
%  \hline
\end{tabular}
\end{table}


\section{Heavy Water Reactors}
This type of reactor was developed in Canada and has since been exported around the world. The use of heavy water allows the use of un-enriched uranium. This appears to be a good feature for proliferation resistance, but the combination of on-line refueling and natural uranium leads to optimal conditions for breeding plutonium for weapons. 
The excellent neutron economy of this design may lead to extended uses in the future. For example used LWR fuel could be used as fuel in a heavy water reactor.
\begin{table}[!h]
\begin{tabular}{c|c}
  Advantages & Disadvantages \\
  \hline
  Natural uranium fuel & Expensive Moderator \\
  On-line refueling & Proliferation Risk \\
  High burnup per ore & Low burnup per fuel element \\
  No pressure vessel & Many pressure tubes \\
%  \hline
\end{tabular}
\end{table}

\section{Graphite-Moderated Water-Cooled Reactors}
These reactors are traditionally associated with weapons production. Graphite moderation allows for natural uranium fuel and efficient breeding of plutonium. The RBMK reactor in Chernobyl was not carefully designed to achieve a negative boiling reactivity coefficient, which was part of the cause of that disaster.
\begin{table}[!h]
\begin{tabular}{c|c}
  Advantages & Disadvantages \\
  \hline
  Natural uranium fuel & Large size \\
  On-line refueling & Proliferation Risk \\
  & Potential for positive reactivity coefficients \\
%  \hline
\end{tabular}
\end{table}

\section{Gas Cooled Reactors}

\subsection{MAGNOX Reactors}

\subsection{Pebble Bed Reactors}

\subsection{Prismatic Block Reactors}

\subsection{Gas-Cooled Fast Reactors}
\begin{table}[!h]
\begin{tabular}{c|c}
  Advantages & Disadvantages \\
  \hline
  Single-phase coolant & High pumping power \\
  Direct cycle energy conversion & High pressure\\
   & Coolant leakage \\
  High temperature & Risk of corrosion \\
  Little absorption in coolant & Low heat capacity\\
  High breeding ratio & Low heat transfer coefficient\\
%  \hline
\end{tabular}
\end{table}


\section{Liquid Metal Cooled Reactors}

\subsection{Sodium(-Potassium) Cooled Fast Reactors}

\subsection{Lead(-Bismuth) Fast Reactors}


\section{Molten Salt Reactors}

