\documentclass[]{book}
%\usepackage{standalone} %you could combine documents this way 
\usepackage{booktabs} % tables
\usepackage{graphicx} %allows .png\ and .pdf
%\usepackage{natbib} % I don't think we need this, just chapterbib
\usepackage{chapterbib}
\usepackage{wrapfig} %allows you to wrap text around a figure

  % Smaller margins, fix these later:
\addtolength{\oddsidemargin}{-0.75in}
\addtolength{\evensidemargin}{-1.25in}
\addtolength{\textwidth}{1.75in}%
\addtolength{\topmargin}{-0.5in}
\addtolength{\textheight}{1.25in}

\begin{document}
%\section*{Cover page}
\noindent Authors: Matthew Eklund, Christopher Morrison, and Jaron Senecal\\
Affiliation: Rensselaer Polytechnic Institute \\
Last updated: \today


\section*{Disclaimer}
This document is for informative and educational purposes only and is incomplete.  
The views and opinions expressed in this book do not necessarily represent those of Rensselaer Polytechnic Institute. 
Additional information will be added throughout the year. Don't sue us, please.


\section*{Preface}
%Well, it's simple...
Teaching college students how to solve equations is easy to do. But teaching students how to design systems is more complex, mostly because it is less regimented. From our experience as graduate students in nuclear engineering we feel that the process of designing reactors has not been adequately covered by the existing curriculum. Perhaps this is due to the fact that reactor designs seem limited to PWRs and new construction is quite rare. Our hope is that this book will help prepare young engineers to design the reactors of tomorrow. (Our hope is that we are among those engineers). Furthermore, all existing nuclear reactors were designed about fifty years ago, so our goal is to learn from those who were there when it happened. Perhaps this book can be used as the basis for a course in reactor design.

\begingroup
\let\cleardoublepage\clearpage
\tableofcontents
\endgroup

\include{content1}

%Chapter 2
\chapter{How Did We Get Here? The History of Nuclear Energy}
In the field of nuclear technology, change is difficult. Developing materials that can withstand high temperatures and intense radiation is a timely, laborious process. 
Proving new technologies and demonstrating extreme reliability of parts  in any foreseeable environment is costly.
Imagining and protecting against extremely low-probability events is even more difficult for new ideas with unknown failure modes.

All of these hurdles and more must be overcome in order implement new reactor designs. At the time of this writing, it seems that these hurdles are so large in the United States that the LWRs will be the only design ever built here.
Indeed, looking back on the history of nuclear power, no reactor design has ever successfully reached market that was not driven primarily by the government. In other words, the benefits of advanced reactor designs are not enough to warrant the monumental investment from industry that would be needed to develop new technology. This is greatly exacerbated by the decades-long research and development time frame.

\section{World War II}
Fission was discovered in December 1938. It's potential was immediately realized and the race to develop the ultimate bomb began. Uranium bombs require enrichment facilities, but plutonium bombs could be fueled by reprocessed fuel from a reactor. Thus the mission for the first reactors was anything that could produce plutonium. The key attributes were the use of natural uranium and quick production time. Safety was established by siting the reactor in an uninhabited area of Washington state and following good engineering practices. Graphite moderated, water cooled reactors were the clear choice because heavy water was too scarce.
%something about uranium metal being the easiest to work with.

Following the war plutonium production was still a key objective so further breeder reactors were built at Hanford. Heavy water reactors were constructed at the Savannah River Site in South Carolina. All of the energy produced from fission was simply dumped into rivers, rather than used for generating electricity.

%Verify, get sources
%In Europe, graphite reactors were also used for breeder plutonium for weapons. The necessary modifications were made to 
% http://www.world-nuclear.org/information-library/current-and-future-generation/outline-history-of-nuclear-energy.aspx

In 1951 the EBR-I reactor demonstrated that nuclear reactors could be used to generate electricity. %what was the main thrust behind building this reactor?
In 1954, Russia began operation of the first nuclear reactor that was connected to an electrical generator with a rating of 5MWe.
Similarly, without sufficient enrichment capabilities, the United Kingdom used natural uranium metal, gas-cooled, graphite moderated reactors. The first of these came on line in 1956.
France followed a similar path as the UK with graphite moderated, gas cooled reactors. 
Canada completely avoided the path to enrichment and utilized their heavy water capabilities to create reactors. 

\section{Naval Reactors}
In the 1950s Admiral Hyman Rickover oversaw the development of pressurized water reactors. In order to fit a reactor into a submarine, highly enriched fuel and light water were used. In 1954 the USS Nautilus was launched. Zirconium purification was developed to support the naval reactors program---previously zirconium was unsuitable due to Hafnium impurities. 

With the research and development covered by the naval reactors program, light water reactors were now feasible for use by electrical utilities. In 1957 the Shippingport 60MWe reactor began operation as the first commercial nuclear reactor in the United States. %world?

\section{Design Evolution}
From the starting point of light water reactors, technology steadily improved performance without any major design changes. Most of these developments involve enhancing fuel performance. 
Various alloying elements were added to zirconium to improve corrosion resistance. This was essential for increasing the amount of time the cladding could survive in the core. 
In 1977 when President Carter banned reprocessing of fuel, reaching higher burnup became the key goal for fuel management. Until that point it had been understood by the industry that fuel would be reprocessed by the Atomic Energy Commission.

Controlling the grain size of sintered UO$_2$ allows for mitigating swelling and better retention of fission products. Creating a dished shape allows the pellets to accommodate thermal expansion and reduce stress on the cladding.  

% burnable absorbers
% chemical shim
% mention BWRs

\section{Growth of Safety Requirements}
From the very beginning, safety has been integral to the development of nuclear reactors. At the Chicago pile, Fermi stationed a Super-Critical Reactor Ax Man to drop boron into the reactor if anything should go wrong.
The plutonium production reactors were sited at Hanford to minimize the impact of a catastrophic failure.
Admiral Rickover demanded that those designing the submarines took as much care as if their sons were aboard the ships.

As an extra precaution, the Shippingport Reactor was constructed with a containment building in case of a reactor meltdown or malfunction. All of the subsequent reactors followed suit. 
Initial safety considerations followed the idea of the maximum credible accident. Initially this was a reactivity insertion. Commercial designs prevented withdrawing control rods too rapidly, so the next scenario to worry about was a Loss Of Coolant Accident (LOCA).
% http://users.owt.com/smsrpm/nksafe/sixties.html
After 1966, the Emergency Core Cooling System (ECCS) began to be viewed as necessary in order to prevent disastrous consequences of a LOCA.  ``it has certainly obtained the bulk of the resources expended in nuclear reactor safety research.''\cite{Safety_website}

In 1975 the Nuclear Regulatory Commission (the successor of the Atomic Energy Commission) published the Reactor Safety Study (WASH-1400). This report used Probabilistic Risk Assessment to estimate the risk that nuclear reactors pose to public health. The report showed that human error and small-break LOCAs were larger concerns than the maximum credible accidents that had been previously fixated on.

The 1979 accident at Three Mile Island unit 2 served as a wake-up call to the nuclear industry that serious accidents could occur. The importance of a safety culture became apparent. Human error finally became a major focus of safety. On the engineering side, better instrumentation and presentation in the control room were now clearly necessary

\section{Other Reactor Designs}
Up to this point in the chapter only a few reactor types have been discussed. This is because nuclear technology has been inextricably linked to government research and development for military use. 
Graphite reactors cooled with either water or gas stemmed from plutonium production efforts as did heavy water reactors. %It should be noted that the Canadian government has foresworn enrichment of uranium and demonstrates no desire to produce plutonium for weapons.
Light water reactors grew out of naval propulsion programs. 

Given that the goals of commercial power plants differ from the military goals that the reactor designs grew out of, it is unlikely that these designs are optimal for power production. Furthermore, the perspective taken on issues of safety has significantly changed in the 50 years since the reactors began operation. 

To be fair, many reactor designs have been created and prototyped. Gas-cooled reactors, sodium-cooled fast reactors, organic-moderated reactors, molten-salt reactors, lead(-bismuth)-cooled reactors have all been operated. But they have all been discontinued because of various setbacks and challenges. Could it be that given a government R\&D push like LWRs experienced, that these advanced reactor designs would out-perform the current standard?

Perhaps the additional impetus will come from shortages of U-235. SFRs have received considerable attention throughout the years because they can accomplish a mission that thermal reactors cannot: breeding their own fuel.
But that goal has become less important as uranium ore deposits continue to be discovered.


%%%%%%%%%%%%%%%%%%%%%%%%%%
\bibliographystyle{ieeetr}
\bibliography{bib2}

\chapter{How to Begin}
This part might go in Chapter 1???

\section{Design Progression}
While it may be tempting to attack a design problem with all of the computational tools at your disposal, that may not be the best approach. Informed decisions at the beginning of the project reduce the amount of work that you and your computer will have to do. 
In general, low-fidelity methods or formulations are entirely satisfactory at the start. Simple calculations also tend to provide understanding and intuition that will be useful as the design process continues.

\begin{figure}[!hb]
  \label{fig:linear_design}
  \centering
  \includegraphics[width=0.75\textwidth]{graphics/design_process_big.png}
  \caption{Simplified design process. (Graphic from http://ahmackenziedesign.com/faq/ )}
\end{figure}

As the design matures, so should the tools being used. Think of computer codes as sieves. Low-fidelity codes will filter out designs that obviously won't work and more sophisticated tools may be needed to decide between closely competing design alternatives.

\begin{wrapfigure}{R}{0.375\textwidth}
  \label{fig:cluttered_design}
  \centering
  \includegraphics[width=0.35\textwidth]{graphics/GeneralDesignProcess_cluttered.pdf}
  \caption{Realistic design process.}
\end{wrapfigure}

That being said, the ultimate goal is to reduce the time needed to generate a satisfactory design. There is no reason to spend more time on a less accurate method. For example, spending an hour to calculate the four-factor formula by hand with analytical methods is not as wise as waiting 15 minutes for the results from a Monte Carlo simulation. So the rough-cut tools you use should also be fast and easy to use.

While it is easier to describe a linear design process, that is usually not how it works in practice. Initially there will likely be several candidate designs which should be investigated in parallel. The number of competing designs will decrease as time goes on. 
Furthermore, design textbooks generally present the design process as a chronological journey through a serious of tasks (Figure~\ref{fig:linear_design}). But this is not accurate, because design is an iterative process. So then the flow charts are given arrows for every point to every preceding point, as in Figure~\ref{fig:cluttered_design}. 
With such a tangled picture in mind, we wish to emphasize from the outset that the design process will involve regular backtracking, especially since ideas are not generated only during brainstorming sessions.
The important thing is keep moving and maintain a willingness to backtrack, modify, and change your design. Trying to design an optimal reactor from the outset will certainly lead to slow going. So do your best to stay out of rabbit holes at the beginning and leave the details for later.


\section{Initial Decisions}
Decisions made at the beginning of the design process have a bigger effect on life cycle costs than decisions made toward the end as shown in Figure~\ref{fig:life_cycle_cost}. 
As time goes on, certain design commitments have been made and retroactive actions are more costly than designing it right the first time.
Clearly, proper engineering design which takes full account of all foreseeable problems will greatly reduce the cost of the project.
Furthermore, the front end of the design phase is also important. So it is worthwhile to carefully examine all of the options available at the beginning of the project. 

Some decisions can be made immediately based on engineering knowledge and an understanding of the design goals. Two of the most generic questions are What enrichment is acceptable? and What energy spectrum should the reactor have? The constraint of using natural uranium will quickly suggest a thermal reactor with only a few options for moderators. The desire to breed plutonium will dictate a high-neutron energy spectrum. 

\begin{figure}[!hbp]
  \label{fig:life_cycle_cost}
  \centering
  \includegraphics[width=0.70\textwidth]{graphics/life_cycle_cost_chart.png}
  \caption{Commitment, system-specific knowledge, and cost (source: B.S. Blanchard \& W.J. Fabrycky, Systems Engineering and Analysis, 3rd Ed., Prentice Hall, 1998, Figure 2.11).}
  %http://www.sole.or.kr/Download/SOLEtech/SOLEtech(Volume%204.12).htm
\end{figure}

After the obvious choices have been made, but without accidentally eliminating any viable choices, there are a few fundamental decisions to make. The choice of moderator (if needed), coolant, fuel isotope, and chemical fuel form are the fundamental features around which the reactor will be designed. These decisions do not require any computer simulations if the correct figures of merit are identified and used. 

Often selection is not straightforward because the materials are better at some things and worse in other respects. In cases like these, the decision is which set of issues you want to deal with.
The way each issue is weighted will influence the decision. For example, if low R\&D requirements is a priority, UO$_2$ will be selected and the poor thermal conductivity will have to be dealt with. 
Dichotomies like this help explain why there remains such a diversity of opinions and research efforts.
In other words, there may be more than one acceptable answer, but it is important to explain why the selection was made.

% Chris will probably want to expand this part quite a bit

\subsection{Coolant Selection}
Nuclear engineers often think of neutronic design as the critical (pun intended) part of the reactor system. However, neutron transport is actually a smaller part of the design than one would think.
In any reactor that produces power, cooling is essential. When the goal is producing electricity, managing the working fluid is the key to reliability and efficiency. Thus we begin with a discussion of coolants.

The first question is, Is this a thermal spectrum reactor? If the answer is no, many low-Z coolants are automatically eliminated from consideration.
Next, the desired operating temperatures need to be assigned ballpark numbers. This is determined by estimating the capabilities of heat exchangers, turbo-machinery, and structural and piping materials. You can prescribe these limits later if necessary, remember design is an iterative process. 

The following coolant properties are desirable for any reactor. \emph{High heat capacity}. This allows the coolant to absorb more energy per change in temperature. \emph{Non-corrosive}. Whatever the coolant contacts, or has the potential to contact, it should be compatible with. If the coolant corrodes the cladding, it will limit the fuel cycle length and sabotage the fuel utilization. \emph{Radiation stability} The coolant in the core will experience high radiation fields and radiolysis and neutron activation will occur. The coolant should be selected so that these phenomena are minimized. \emph{Low cross section} Neutrons absorbed in the coolant or coolant impurities will diminish the performance of the reactor \cite{Hausner}.

%water
% liquid metal
% gas
% salts
% other

\subsubsection{Comparison of Liquid and Gaseous Coolants}
The phase of the coolant is an important consideration. Both liquid and gaseous coolants have their advantages and disadvantages.
If gaseous coolant is chosen, the chemically inert noble gases become attractive options.

\begin{table}[!ht]
\begin{tabular}{c|c|c|c}
  Qualtiy & Liquid & Two-Phase & Gaseous \\
  \hline
  Heat Capacity & High & High & Low \\
  Phase Change & Limiting condition & Desirable & Not an issue \\
  Heat Transfer Coefficient & High & Very high & Low \\
  Natural Circulation & Good & Very good & Poor \\
  Pumping Power & Low/good & Low/good & High/bad\\
  Neutron Absorption & High & Less & Low\\
  Direct Cycle & No & Yes, Rankine & Yes, Brayton\\
  Low Pressure & Maybe & No & No\\
  %
  \hline
\end{tabular}
\caption{Phase of coolant}
\end{table}

%\subsubsection{Comparison of Molten Salts and Liquid Metals}
%\subsubsection{Comparison of Moderating Coolants}
% or:
%\subsubsection{Comparison of Liquid Coolants}


\subsection{Moderator Selection}
For thermal reactors, a low-Z element is needed to slow down the fission neutrons to low energy levels where fission is more likely to occur. It must also do this without absorbing too many neutrons. Hydrogen is an obvious choice because of its ability to stop a neutron completely with only one scattering interaction and its low mean free path. Deuterium comes to mind next because it absorbs neutrons 1000 times less. Graphite is a desirable moderator because it can withstand extremely high temperatures.
%reproduce Duderstadt discussion and table

% \subsection{Figures of Merit}

\subsection{Fuel Selection}
The requirements of nuclear fuel are very demanding due to the environment it must withstand. Hausner \cite{Hausner}
lists the following requirements for nuclear fuel
\begin{itemize}
\item[1.] It must be able to tolerant radiation damage
\item[2.] It must not corrode upon contact with the cladding or coolant.
\item[3.] It or its impurities must not absorb too many neutrons
\item[4a.] It must be able to withstand the temperature gradient caused by the heat it generates
\item[4b.] It should have high thermal conductivity (or appropriate dimensions) in order to transmit its heat to the coolant
\item[5.] It must be able to withstand the thermal cycles that it will experience during operation
\item[6.] It must be able to withstand the mechanical loads placed upon it
\item[7.] It must be inexpensive
\item[8.] It should lend itself to the fuel cycle in which it will be used, e.g. easy to reprocess
\end{itemize} 

Also it is generally preferable to have a high density in order to ease requirements for enrichment.

\subsection{Cladding Selection}
Material selection for cladding is important because it is the first line of defense for retaining radioactive material.
Even for reactor designs with no cladding, e.g. a molten salt-fueled reactor, it is necessary to select compatible and durable materials to contain the nuclear material and working fluid.
Selecting reliable cladding and piping materials allows the reactor to run smoothly with fewer interruptions. The primary requirement is for the cladding to be chemically compatible with both the fuel and the coolant. Superior corrosion performance is the foremost concern.

The cladding material must also be capable of enduring an extreme environment of neutron, gamma, and fission product irradiation and high temperatures. It must also maintain its strength and ductility while absorbing as few as possible neutrons. 
After all this, it would be preferable if the material was cheap and easy to manufacture.

\subsection{Secondary Circuit Working Fluid Selection}
For power reactors that do not use a direct cycle, a working fluid in the secondary system must be selected. Generally the working fluid for energy conversion can be selected after the primary coolant has been determined. 
However, the potential secondary fluids may weigh on the decision of the primary fluid.
For example, sodium coolant may be disfavored because of its incompatibility with water. 
But for most other coolants the choice is not as important. 

Water is by far the most used working fluid for nuclear reactors. Even sodium-cooled reactors have converted energy with a steam system (making the complexity of intermediate loops and pipe-in-pipe heat exchangers necessary). 
The widespread use of steam turbines has made that power conversion system quite familiar.

Several advanced reactor concepts call for the use of the Brayton cycle with Helium or super-critical CO$_2$ as the working fluid. These systems can (must) work at higher temperatures and they achieve higher efficiency.

% talk about why a Mercury Rankine cycle is not very good.

\section{Fuel Geometric Specification}
After these have been determined, the basic core geometry comes next. 
At this point an optimized design is unlikely and even unneeded. More likely each analysis will provide the limits of a feasible design space. 

\begin{figure}[!hbt]
  \label{fig:fuel_geom}
  \centering
  \includegraphics[width=0.70\textwidth]{graphics/SAperVOL.pdf}
  \caption{Surface area per volume is a function of fuel layout.}
\end{figure}

There are several options for fuel layout: blocks, pins, plates, pebbles, and fluid.
Blocks have the smallest surface area for heat transfer, whereas plates and small pebbles/particles have more surface area per volume, see Figure~\ref{fig:fuel_geom}. 
%
%Figure about dispersion vs solid fuel meat
%
For optimal neutronics performance, the percentage of cladding material must be reduced. Thus large fuel pins are favored.
From the heat transfer perspective the surface area of the fuel per volume should be maximized. This pushes the design towards plate fuel.
In the extreme case, the fuel is the working fluid and the reactor core does not need a heat transfer surface.
These competing requirements will lead to trade-offs and constraints.

Within these fuel types, the fuel meat can be composed of either uniform fuel (e.g. a pellet of UN) or some type of composite (e.g. TRISO fuel in a graphite matrix).
Solid fuels allow for smaller reactors cores or lower enrichment.
Dispersion fuels such as TRISO particles in a graphite matrix can combine favorable material properties. Fissile material with a low thermal conductivity (such as UO$_2$) can be embedded in a matrix with a high thermal conductivity. 
This enables efficient heat transfer to the working fluid and avoids excessive temperatures in the fissile material.
The primary drawback of composite fuels is that the fuel volume fraction in the reactor core is reduced. For graphite reactors this is not a problem because the moderator to fuel ratio is so large anyway.

% Fuel pins increase centerline temperatures, but minimize the amount of cladding required, thus the neutronics expert prefers this option. The heat transfer expert prefers a design that tends towards a plate with more surface area per volume. 

Fluid fuel has only been used in the Molten Salt Reactor Experiment. 
Fluid fuels are not used primarily because of the high temperatures required (which is very taxing for materials) and the potential mobility of fuel.
In an accident, the worst case is that the nuclear fuel and radioactive material can not be accounted for. Containing radioactivity is the primary safety mandate for nuclear reactors and fluid fuels make this more difficult.

At this stage of the design, the unit cell is typically the geometry of interest. 
The representative fuel pin with its surrounding coolant channel is taken to be part of an infinite lattice. This is modeled by using reflected boundary conditions.
%Add notes about computational set up: MC vs Deterministic, multiphysics or separated, reduced order models.

% How can we predict limitations from accidents at this point? We don't want to start with no margin.

\section{Assembly Design}
Where does this section go?

Fuel pins are arranged into assemblies in order to resist buckling and simplify fuel handling.
To be compatible with existing reactors, the size of the assembly must be constant.
So an 11-pin-by-11-pin assembly can replace a 10x10 assembly if it has the same size.

\section{Core Geometry}
Neutron leakage from the reactor core can have a significant reactivity effect.
This penalty is minimized by making the core as large as possible---this is one economy of scale. For a fixed power rating, creating a reactor core in the shape of a sphere is optimal for reducing neutron leakage.
Such a geometry is often impractical because it precludes (or greatly complicates) fuel shuffling. It is also difficult to construct.
For this reason, reactor cores are generally assembled in the shape of cylinder.

\begin{table}
\centering
\large
\caption{Geometric Buckling Factors. $\nu_0=2.405$. Tildes denote extrapolated lengths. Geometric buckling is additive whereas the peaking factors and flux profiles are multiplicative. (Adapted from Duderstadt.)}
\setstretch{2.0}
\begin{tabular}{lccc}
\hline
Geometry & Geometric Buckling $B_g^2$ & Flux Profile & Peak/Average \\
\hline
Slab & $\left( \frac{\pi}{\tilde{a}} \right)^2$ & $\cos \frac{\pi x}{\tilde{a}}$ & 1.571\\
Infinite Cylinder & $\left( \frac{\nu _0}{\tilde{R}} \right)^2$ & $J_0 \frac{\nu_0 r}{\tilde{R}}$ & 2.316\\
Sphere & $\left( \frac{\pi}{\tilde{R}} \right)^2$ & $r^{-1}\sin \frac{\pi r}{\tilde{R}}$ & \\
Finite Cylinder & $\left( \frac{\pi}{\tilde{H}} \right)^2 + \left( \frac{\nu _0}{\tilde{R}}
     \right)^2$    & $J_0 \frac{\nu_0 r}{\tilde{R}} \cos \frac{\pi z}{\tilde{H}}$ & 3.638\\
\hline

\end{tabular}
\end{table}
% make the rows taller

Sodium cooled reactors rely on increased leakage to mitigate the positive sodium void coefficient of reactivity that comes from reduced absorptions in the coolant.
This is accomplished with a ``pancake core'' design with a radius larger than the height.

%We need some figures in this section...

Annular cores can be used to mitigate power peaking...

The height of the core is usually limited by thermal hydraulic considerations. 
In order to limit the pressure drop and temperature change across the core the height should not be too large.


%%%%%%%%%%%%%%%%%%%%%%%%%%

\begingroup
\let\cleardoublepage\clearpage

\bibliographystyle{ieeetr}
\bibliography{bib3}

\endgroup


\chapter{Learning from Other Designs}
In the process of learning about nuclear engineering one undoubtedly learns a great deal about existing reactor designs. On one hand the designer should start a project with a clean slate, but on the other hand knowledge in memory can greatly accelerate the design process.
Trends, heuristics, and limitations are useful as long as the designer does not prematurely eliminate design possibilities.

In this chapter we will discuss types of nuclear reactors that have been built. The strengths and shortcomings will be listed, although not exhaustively. 
The purpose of this chapter is not to promote one design over another, because different constraints, priorities, and goals will favor different designs.

\section{Light Water Reactors}
Light water reactors are by far the most common reactor in the world. 
\begin{table}[!h]
\begin{tabular}{c|c}
  Advantages & Disadvantages \\
  \hline
  Widely used technology &  \\
  Inexpensive coolant & Corrosion issues \\
   & Requires enriched fuel \\
%  \hline
\end{tabular}
\end{table}

\subsection{Pressurized Water Reactors}
Pressurized water reactors are the more common type of light water reactor. The water must be kept high pressure to preclude boiling inside the primary coolant loop.

\subsection{Boiling Water Reactors}
Boiling Water Reactors aim for improved efficiency by using a direct power conversion cycle. However, boiling in the reactor core leads to a variety of challenges. 
\begin{table}[!h]
\begin{tabular}{c|c}
  Advantages & Disadvantages \\
  \hline
  Improved Efficiency & Two-phase flow \\
  Load following & Unstable at low power\\
   & Radioactivity in the steam turbine\\
%  \hline
\end{tabular}
\end{table}


\section{Heavy Water Reactors}
This type of reactor was developed in Canada and has since been exported around the world. The use of heavy water allows the use of un-enriched uranium. This appears to be a good feature for proliferation resistance, but the combination of on-line refueling and natural uranium leads to optimal conditions for breeding plutonium for weapons. 
The excellent neutron economy of this design may lead to extended uses in the future. For example used LWR fuel could be used as fuel in a heavy water reactor.
\begin{table}[!h]
\begin{tabular}{c|c}
  Advantages & Disadvantages \\
  \hline
  Natural uranium fuel & Expensive Moderator \\
  On-line refueling & Proliferation Risk \\
  High burnup per ore & Low burnup per fuel element \\
  No pressure vessel & Many pressure tubes \\
%  \hline
\end{tabular}
\end{table}

\section{Graphite-Moderated Water-Cooled Reactors}
These reactors are traditionally associated with weapons production. Graphite moderation allows for natural uranium fuel and efficient breeding of plutonium. The RBMK reactor in Chernobyl was not carefully designed to achieve a negative boiling reactivity coefficient, which was part of the cause of that disaster.
\begin{table}[!h]
\begin{tabular}{c|c}
  Advantages & Disadvantages \\
  \hline
  Natural uranium fuel & Large size \\
  On-line refueling & Proliferation Risk \\
  & Potential for positive reactivity coefficients \\
%  \hline
\end{tabular}
\end{table}

\section{Gas Cooled Reactors}

\subsection{MAGNOX Reactors}

\subsection{Prismatic Block Reactors}
The fuel for these reactors is composed of millimeter-scale TRISO particles. A kernel of UO$_2$ is surrounded by several layers of graphite and silicon carbide which act as containers for fission products.
The TRISO particles are interspersed in a graphite matrix. This fuel block has various holes for coolant flow and fuel handling. Fuel blocks are stacked to compose the reactor core. 
The large mass of graphite has a very large heat capacity which can absorb the energy of almost any transient. 

Graphite reactors do have some safety issues, however. First, dislocations to the graphite atomic matrix build up with neutron fluence. The reactor must be taken to higher temperatures periodically in order to anneal the graphite and release the energy stored in these imperfections. %cite windscale
Also, there is still some debate about the flammability of graphite.
\begin{table}[!h] \label{tab:Prismatic}
\begin{tabular}{c|c}
  Advantages & Disadvantages \\
  \hline
  High heat capacity & Requires annealing\\
  High temperature ceramic & Flammable?\\
  Low absorption moderator & Large core\\
  Multiple layers around fuel & Complex manufacturing\\
  Allows high output temperature & High pumping power\\
  & Complex/large spent fuel waste\\
%  \hline
\end{tabular}
\end{table}

\subsection{Pebble Bed Reactors}
Pebble bed reactors share many features with prismatic block reactors. However, by using many fuel pebbles, better fuel economy can be achieved. %factor of two for continuous reloading, right?
Construction of the reactor core is greatly simplified: it is essentially a can with guide tubes and a bottom nozzle. 
\begin{table}[!h]
\begin{tabular}{c|c}
  Advantages & Disadvantages \\
  \hline
  Increased burnup & Complex fuel manufacturing\\
  Replaceable fuel elements & Greater likelihood of individual failure\\
   & Complex, indeterminate geometry\\
%  \hline
\end{tabular}
\caption{See Table~\ref{tab:Prismatic} for the pros and cons common to graphite reactors.}
\end{table}

\subsection{Gas-Cooled Fast Reactors}
Gas cooled fast reactors share several features with gas-cooled thermal reactors, but the lack of a large mass of graphite moderator greatly reduces the heat capacity of the reactor core. Without out this sink for energy during accidents, passive safety is much harder to attain.

The original motivation for gas cooled-fast reactors was improved neutron economy due to decreased absorption in the coolant. 
\begin{table}[!h]
\begin{tabular}{c|c}
  Advantages & Disadvantages \\
  \hline
  Single-phase coolant & High pumping power \\
  Direct cycle energy conversion & High pressure\\
   & Coolant leakage \\
  High temperature & Risk of corrosion \\
  Little absorption in coolant & Low heat capacity\\
  High breeding ratio & Low heat transfer coefficient\\
%  \hline
\end{tabular}
\end{table}


\section{Liquid Metal Cooled Reactors}
Liquid metal coolants have been considered for fast spectrum reactors since the earliest days of nuclear energy. It was the NaK-cooled Experimental Breeder Reactor (EBR-I) that first produced electricity from nuclear energy.
Liquid metals are excellent heat transfer media and also allow for the benefits of a fast-spectrum.

\subsection{Sodium(-Potassium) Cooled Fast Reactors}
Sodium is an excellent heat transfer medium, but its melting temperature is 98$^{\circ}$C. Either the coolant loops must be heated continually during shutdown, or potassium must be added to lower the melting temperature. Both metals are highly flammable. Potassium absorbs more neutrons.
\begin{table}[!h]
\begin{tabular}{c|c}
  Advantages & Disadvantages \\
  \hline
  Good heat transfer & Incompatible with air and water\\
  Fast neutron spectrum & shorter neutron lifetime\\
  Passive safety demonstrated & Sodium activated by neutrons \\
  Electronic pumping & opaque coolant\\
  & Sodium: solid at room temperature\\
  High temperature & Intermediate coolant loop\\
%  \hline
\end{tabular}
\end{table}

\subsection{Lead(-Bismuth) Fast Reactors}
Bismuth is added to lead to decrease the melting point. Lead does not react with water, air, or CO$_2$, which means intermediate cooling loops are unnecessary. 
There is less experience with lead-cooled reactors. Some Russian submarines had lead-cooled reactors, but there were eventually decommissioned.
Corrosion issues can be managed by controlling the oxygen content of the coolant. The protective oxide layer will remain intact if the coolant flow rate does not exceed about 2m/s.
\begin{table}[!h]
\begin{tabular}{c|c}
  Advantages & Disadvantages \\
  \hline
  Good heat transfer & Melting point 327$^{\circ}$C\\
  Insignificant moderation & Inelastic scattering\\
  Lead: does not absorb & Bismuth: creates Polonium\\
  Inexpensive material & very heavy\\ 
   & opaque coolant\\
  High boiling point & Erosion limit on flow rate\\
%  \hline
\end{tabular}
\end{table}


\section{Molten Salt Reactors}
Molten salt reactors have received much attention in recent years, but have little operating experience. 




\chapter{Design for Safety and Operations}
%feel free to move this to chapter 4.
%We should have a better naming system, so that it's easier to rearrange chapters

After the initial design has brought the reactor concept to a feasible design space it is time to begin a more thorough analysis.
At this point computational speed is not as essential because modifications to the design are smaller than at the initial stages.

\section{Burnup Calculations and Fuel Life}
In order to maximize capacity factors and profits, longer fuel cycles are desirable. 
The initial design activities delineated a feasible design space where the configuration would be critical. In this phase of analysis, depletion calculations must be run to estimate the life of the fuel. The fuel composition will be adjusted to attain the required fuel life.

%Talk about Xenon override

\section{Burnable Poisons, Reactivity Control, and Power Peaking Factors}
%burnup-> reactivity control

The pin generating the most power will be the first to fail. Thus safety considerations usually focus on this limiting region of the core. For a cylindrical core, the power at the center of the core will be 3.64 times larger than the average power. 
Ideally, all of the fuel elements would operate at the average value so that the reactor could operate at a power that is 3.64 times higher.

To flatten the power distribution, a variety of methods are available. 
The most obvious method being control rods. However, that is not the most desirable because it distorts the axial power profile.
Burnable poisons are widely used, varied enrichments are also common. 
Batch refueling adds the extra freedom of arranging fuel assemblies with different burnups. 
Removable absorbers (e.g. WABA) can be varied from batch to batch.


\section{Control Rod Layout}
Analytical calculations of control rod worth simplify the control rods into one central element. This may be a useful method for making estimates, but it is certainly not the appropriate way to design the control rods.
Ever since the SL-1 accident, it has been accepted that no single control rod/device should have enough reactivity to bring the reactor critical. Thus if one rod fails, the others can still shut down the reactor.

% Control-rod shadowing (self-shielding)

% \section{Control Rod Programs}

\section{Safety Analysis}
%Check PRA notes to get the safety analysis progression correct

Safety analysis for nuclear reactors is a strange combination of quantitative logic and whimsical imagination.
On one hand systems and components are designed (and thoroughly tested in zealous quality assurance programs) to perform as intended over a wide range of conditions.
On the other hand, accidents are postulated and analyzed where those systems fail. 
In the nuclear industry it is not satisfactory to design a reactor that will never melt down;
designers must also show that if it did, the danger to the environment is minimal.

At first blush this seems absurd because it flies in the face of physics. 
But it is actually an effective approach for obtaining high levels of safety.
There is no way to foresee every possible initiating event or mode of failure, so simply assuming a component failure helps mitigate the unforeseeable. 
For example, the failure of a Reactor Pressure Vessel is thought of as extremely unlikely due to the sound engineering principles codified in the ASME Boiler and Pressure Vessel Code.
However, at the Davis-Besse plant in 2002 severe corrosion of the vessel head was discovered.
If it had not been noticed, there could have been a severe loss of coolant accident.
Incidents like this show how certain postulated accidents can be more likely than expected and they do in fact merit analysis.

Distrust of individual systems and components leads to redundancy referred to as Defense in Depth. DID provides multiple layers of protection to maintain safety even when one layer (or several) fail. Redundant safety measures are complimented by diverse systems. 
Diversity allows the intended function to be fulfilled in all cases because no initiating event disables all of the safety systems.
Defense in Depth typically proceeds along these lines: fission products are contained in the fuel pellet, the fuel pellet is encased in cladding, the fuel assemblies are enclosed in the RPV, the RPV is inside a containment building, and if all of these rupture there is a large distance between the reactor and any populated area.

A nuclear reactor operating at steady state is very safe. Failures due to creep and corrosion are easy to design for.
The situations that really tax the nuclear reactor are when it deviates from steady state.
Initially, the idea behind nuclear reactor safety was to survive the maximum credible accident.
Since then, probabilistic risk analysis has expanded the view of safety to include smaller events that are more likely to occur. %turned attention to dealing with the most likely accident. 

The standard accident scenarios include: loss of coolant accidents (large break or small break), reactivity insertion accidents (control rod ejection, boron dilution accident, moderator overcooling, etc.), loss of heat exchanger, station blackout, loss of flow accident (blockages, pump failures).

Traditionally, the most analyzed accident scenarios are the Reactivity Insertion Accident (RIA) and the large break Loss of Coolant Accident (LOCA). 
For a water-cooled reactor, the LOCA scenario is particularly troubling because the loss of pressure in the primary circuit leads to the water flashing to steam and the core being uncovered.

New reactor designs will be susceptible to different accidents and failures. For this reason, it is imperative to be as imaginative as possible at this stage.
(Try thinking of ways to compromise your reactor design if you were a nefarious person.)
The earlier a path to failure is identified, the easier it will be to deal with.
Good designs are simple and inherently avoid as many mechanisms for failures/accidents as possible.

\subsection{Loss of Coolant Accident}
This severe accident looks different in different reactor types. For a Light Water Reactor, a sudden loss in pressure may uncover the core when the water flashes to steam.
For a gas cooled reactor, the heat transfer is much less effective with a loss of pressure.
For a liquid metal cooled reactor, the coolant is not pressurized so it will not boil vaporize like water. But voiding in the core can be a positive reactivity, which merits investigation.

\subsection{Reactivity Insertion Accident}
Usually the hypothetical scenario for an RIA is a Rod Ejection Accident. Hypothetically, the high pressure inside a reactor vessel could eject a control rod if there was a seal breach.
This clearly makes no sense in the case of a low pressure metal or salt cooled reactor, but it is good to know how a reactor responds to a step insertion of reactivity.
There are other causes of reactivity insertions (e.g. excess heat removal from the coolant) so this is a class of accident that cannot be neglected.

\subsection{Loss of Flow Accident}
The nuclear reactor continues to generate power after a pump stops working. Even if a scram is initiated immediately, the decay heat is unavoidable.
To ameliorate this problem, flywheels or other arrangements let the coolant wind down over a longer period of time in order to maintain cooling.

\section{Homework Problems}
\begin{enumerate}
  \item Give an example of a) redundant and b) diverse safety systems.
  \item For your reactor design, what is the worst accident scenario of those listed above? 
        Is this ranked differently than it was for LWRs?
        Does your reactor design preclude any of these scenarios?
        Are there other scenarios that should be considered?
\end{enumerate}

\chapter*{Material Properties}
This appendix of material properties should be useful for initial comparison studies. For precise values several resources are recommend in a later section.


\begin{figure}[htbp]
  \centering
  \includegraphics[width=4in]{graphics/thermal-k.png}  
  %\caption{Thermal conductivities}  
\end{figure}



\begin{table}
  \centering
  \caption{Basic Thermal Properties\cite{IAEA_1}}
  \begin{tabular}{lcccc}
    \toprule  
    Material & Conductivity & Density & Melting Point & Thermal Expansion\\ 
    %Heat Capacity,K/W/g
             & W/mK &    g/cm$^3$     &   $^{\circ}$C  &  1E6/$^{\circ}$C \\
    \midrule    
    UO$_2$ & 2.79 & 10.963 & 2800 & 9.8\\
    UN     & 20.9 & 14.42  & 2850 & 7.5\\
    UC     & 23   & 13.63  & 2365 & 10.5\\
    U metal & 31.2 & 19.1  & 1130 & 13.9\\
    U-20Pu-10Zr & 16 & 15.67 & 1155 &17\\
    U$_3$Si$_2$ & 22 & 12.2  & 1700 &16.1 \\
    \bottomrule
  \end{tabular}

\end{table}


A good resource has been provided by the IAEA\cite{IAEA_1}.


%%%%%%%%%%%%%%%%%%%%%%%%%%%%%%
\begingroup
\let\cleardoublepage\clearpage


\bibliographystyle{ieeetr}
\bibliography{matbib}

\endgroup


\end{document}