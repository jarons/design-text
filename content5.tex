\chapter{Design for Safety and Operations}
%feel free to move this to chapter 4.
%We should have a better naming system, so that it's easier to rearrange chapters

After the initial design has brought the reactor concept to a feasible design space it is time to begin a more thorough analysis.
At this point computational speed is not as essential because modifications to the design are smaller than at the initial stages.

\section{Burnup Calculations and Fuel Life}
In order to maximize capacity factors and profits, longer fuel cycles are desirable. 
The initial design activities delineated a feasible design space where the configuration would be critical. In this phase of analysis, depletion calculations must be run to estimate the life of the fuel. The fuel composition will be adjusted to attain the required fuel life.

%Talk about Xenon override

\section{Burnable Poisons, Reactivity Control, and Power Peaking Factors}
%burnup-> reactivity control

The pin generating the most power will be the first to fail. Thus safety considerations usually focus on this limiting region of the core. For a cylindrical core, the power at the center of the core will be 3.64 times larger than the average power. 
Ideally, all of the fuel elements would operate at the average value so that the reactor could operate at a power that is 3.64 times higher.

To flatten the power distribution, a variety of methods are available. 
The most obvious method being control rods. However, that is not the most desirable because it distorts the axial power profile.
Burnable poisons are widely used, varied enrichments are also common. 
Batch refueling adds the extra freedom of arranging fuel assemblies with different burnups. 
Removable absorbers (e.g. WABA) can be varied from batch to batch.


\section{Control Rod Layout}
Analytical calculations of control rod worth simplify the control rods into one central element. This may be a useful method for making estimates, but it is certainly not the appropriate way to design the control rods.
Ever since the SL-1 accident, it has been accepted that no single control rod/device should have enough reactivity to bring the reactor critical. Thus if one rod fails, the others can still shut down the reactor.


% \section{Control Rod Programs}

\section{Safety Analysis}
A nuclear reactor operating at steady state is very safe. Failures due to creep and corrosion are easy to design for.
The situations that really tax the nuclear reactor are when it deviates from steady state.
Initially, the idea behind nuclear reactor safety was to survive the maximum credible accident.
Since then, probabilistic risk analysis has turned attention to dealing with the most likely accident. 

The standard accident scenarios include: loss of coolant accidents (large break or small break), reactivity insertion accidents (control rod ejection, boron dilution accident, moderator overcooling, etc.), loss of heat exchanger, station blackout, loss of flow accident (blockages, pump failures).

Traditionally, the most analyzed accident scenarios are the Reactivity Insertion Accident (RIA) and the large break Loss of Coolant Accident (LOCA). 
For a water-cooled reactor, the LOCA scenario is particularly troubling because the loss of pressure in the primary circuit leads to the water flashing to steam and the core being uncovered.

New reactor designs will be susceptible to different accidents and failures. For this reason, it is imperative to be as imaginative as possible at this stage.
(Try thinking of ways to compromise your reactor design if you were a nefarious person.)
The earlier a path to failure is identified, the easier it will be to deal with.
Good designs are simple and inherently avoid as many mechanisms for failures/accidents as possible.

\subsection{Loss of Coolant Accident}

\subsection{Reactivity Insertion Accident}

\subsection{Loss of Flow Accident}
The nuclear reactor continues to generate power after a pump stops working. Even if a scram is initiated immediately, the decay heat is unavoidable.
To ameliorate this problem, flywheels or other arrangements let the coolant wind down over a longer period of time in order to maintain cooling.

