\documentclass[]{article}
\begin{document}
\section{Preface}
%Well, it's simple...
Teaching college students how to solve equations is easy to do. But teaching students how to design is more complex, mostly because it is less regimented. From our experience as graduate students in nuclear engineering we feel that the process of designing reactors has not been adequately covered by the existing curriculum. Perhaps this is due to the fact that reactor designs seem limited to PWRs and new construction is quite rare. Our hope is that this book will help prepare young engineers to design the reactors of tomorrow. (Our hope is that we are among those engineers). Perhaps this book can be used as the basis for a course in reactor design.

\section{Initial Design}
It all starts with a mission, a set of goals that must be accomplished. There are five types of missions: Commercial power, propulsion, science/medical, military, and political.
\begin{itemize}
  \item Commercial power. Missions typically state requirements on power rating, fuel loading cycle, and operability. Constraints are usually to satisfy regulatory requirements for safety. Design goals include anything from small modular reactors to breed and burn reactors to the familiar behemoth LWR. The idea is usually to maximize profits to the owner/operator.
  \item Propulsion. Nuclear fission provides a large supply of power in a very compact space. At various times the ability to continue almost indefinitely without refueling has been desired for marine, aeronautical, locomotive, and space propulsion. Thus the requirement is to minimize size and weight and provide power on demand.
  \item Science/medical. Small test reactors are desirable for producing irradiating targets. The goal may be to produce isotopes through neutron activation or fission or to test materials under neutron bombardment. Thus the goal is to provide neutron flux to a target. The product of these reactors is either data or isotopes. 
  \item Military. The power density of nuclear is desirable to the military because it eliminates the need for supply chains. (Sub)marine propulsion is the primary application, but there is also interest in land-based power. Cost becomes less of an issue for these applications and where power may be needed at a moment's notice.
  \item Political. Nuclear reactors may be commissioned as a demonstration of a country's technological capabilities. Due to the cost of R\&D and manufacturing, government leaders are often involved in the nuclear industry. Political missions may fulfill long term needs that could not be supported by commercial entities.
\end{itemize}

Given the mission, a list of requirements must be generated. There will be constraints and quantities to be optimized. For example, the customer desires maximum thermal flux, but also expects a safe reactor. This list of requirements and reactor attributes will guide the reactor design process.

Several key attributes are desirable for almost every nuclear reactor
\begin{itemize}
  \item Safety. Nuclear engineers are ethically responsible for designing reactors that will not harm the public.
  \item Economics. Saving money is always good.
  \item Efficient fuel utilization. 
  \item Non-proliferation. This reactor should not be an avenue for your enemies to acquire materials for nuclear weapons or dirty bombs.
  \item
\end{itemize}
There are many attributes that may be desirable or required for a reactor. The mission will dictate which features are necessary, desirable, or unneeded.
\begin{itemize}
  \item Minimizes enrichment requirements
  \item Breeds fissile material from fertile isotopes
  \item Destroys stockpiled fissile materials
  \item Destroys transuranic waste
  \item Minimizes total fissile material in-core inventory
  \item Longer intervals between refueling
  \item On-line refueling
  \item Provides high temperature process heat
  \item Capable of changing power rapidly to follow load demands
  \item Low weight to enable mobility or efficient propulsion
  \item High flux (fast or thermal) on target
\end{itemize}

Nuclear reactor design is multi-disciplinary by nature. Neutronics design affects thermal-hydraulic design and vice versa. Structural and material analysis must also be included. <insert flow-chart graphic>. The coupled effects between disciplines confound the design process, thus design iteration is necessary.

First the design parameters must be selected for each discipline. Pitch and diameter of the fuel pins affects thermal-hydraulics, whereas enrichment is a parameter of the neutronics design. It is necessary to obtain an understanding of how these parameters effect the metrics of interest. 



\end{document}