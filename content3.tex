\chapter{How to Begin}
This part might go in Chapter 1???

\section{Initial Decisions}
Some decisions can be made immediately based on engineering knowledge and an understanding of the design goals. Two of the most generic questions are What enrichment is acceptable? and What energy spectrum should the reactor have? The constraint of using natural uranium will quickly suggest a thermal reactor with only a few options for moderators. The desire to breed plutonium will dictate a high-neutron energy spectrum. 

After the obvious choices have been made, but without accidentally eliminating any viable choices, there are a few fundamental decisions to make. The choice of moderator (if needed), coolant, fuel isotope, and chemical fuel form are the fundamental features around which the reactor will be designed. These decisions do not require any computer simulations if the correct figures of merit are identified and used. 

% Chris will probably want to expand this part quite a bit
% \subsection{Figures of Merit}

While it is easier to describe a linear design process, that is usually not how it works in practice. Initially there will likely be several candidate designs which should be investigated in parallel. The number of competing designs will decrease as time goes on. 
Furthermore, design textbooks generally present the design process as a chronological journey through a serious of tasks. But this is not accurate, because design is an iterative process. So then the flow charts are given arrows for every point to every preceding point. 
With such a tangled picture in mind, we wish to emphasize from the outset that the design process will involve regular backtracking, especially since ideas are not generated only during brainstorming sessions.

After these have been determined, the basic core geometry comes next. Fuel pins increase centerline temperatures, but minimize the amount of cladding required, thus the neutronics expert prefers this option. The heat transfer expert prefers a design that tends towards a plate with more surface area per volume. 
%discuss other options: molten fuel, dispersions, particles
At this point an optimized design is unlikely and even unneeded. More likely each analysis will provide the limits of a feasible design space. 

Insert a figure: flow chart or reactor categorization.

\section{Design Progression}
While it may be tempting to attack a design problem with all of the computational tools at your disposal, that may not be the best approach. Informed decisions at the beginning of the project reduce the amount of work that you and your computer will have to do. 
In general, low-fidelity methods or formulations are entirely satisfactory at the start. Simple calculations also tend to provide understanding and intuition that will be useful as the design process continues.

As the design matures, so should the tools being used. Think of computer codes as sieves. Low-fidelity codes will filter out designs that obviously won't work and more sophisticated tools may be needed to decide between closely competing design alternatives.

That being said, the ultimate goal is to reduce the time needed to generate a satisfactory design. There is no reason to spend more time on a less accurate method. For example, spending an hour to calculate the four-factor formula by hand with analytical methods is not as wise as waiting 15 minutes for the results from a Monte Carlo simulation. So the rough-cut tools you use should also be fast and easy to use.


