%Chapter 2
\chapter{How Did We Get Here? The History of Nuclear Energy}
In the field of nuclear technology, change is difficult. Developing materials that can withstand high temperatures and intense radiation is a timely, laborious process. 
Proving new technologies and demonstrating extreme reliability of parts  in any foreseeable environment is costly.
Imagining and protecting against extremely low-probability events is even more difficult for new ideas with unknown failure modes.

All of these hurdles and more must be overcome in order implement new reactor designs. At the time of this writing, it seems that these hurdles are so large in the United States that the LWRs will be the only design ever built here.
Indeed, looking back on the history of nuclear power, no reactor design has ever successfully reached market that was not driven primarily by the government. In other words, the benefits of advanced reactor designs are not enough to warrant the monumental investment from industry that would be needed to develop new technology. This is greatly exacerbated by the decades-long research and development time frame.

\section{World War II}

\section{Naval Reactors}

\section{Design Evolution}

\section{Growth of Safety Requirements}
